\subsection{セットアップ}
最初に,本レポートの中核をなす細胞モデルを構成する.
このモデルは,Himeoka and Kaneko (2014)のモデル\cite{hk14}をさらに簡単にしたものである.
まず,外部から栄養を取り込み成長する一つの細胞を考える.
この細胞は,外部の栄養分子N(濃度$n$)を取り込み,次の自己触媒反応により細胞内で代謝物X(濃度$x$)を生成する:
\begin{equation}
  \ce{N + X ->[$k$] X + X}.
\end{equation}
ただし,$k$はこの反応の速度定数である.
また,簡単のため,細胞膜における栄養分子の出入りは十分に素早く,細胞内の栄養濃度は細胞外の栄養濃度$n$に等しいと考える.
そのため,ここでは細胞内の栄養分子と細胞外の栄養分子を区別しないことにする.
次に,代謝物は反応速度$\phi$で消費されるとする:
\begin{equation}
  \ce{X ->[$\phi$] $\emptyset$}.
\end{equation}
そして,細胞の成長速度$\mu=V^{-1}dV/dt$($V$は細胞の体積)は代謝物Xの消費速度に比例すると考え,$\mu=\gamma \phi x$とする.
この$\mu$は成分の希釈率も意味する.

このとき,代謝物Xの濃度$x$は,微分方程式
\begin{equation}
  \frac{dx}{dt} = k n x - \phi x - \mu x
\end{equation}
に従う.
これはロジスティック方程式
\begin{equation}
  \frac{dx}{dt} = \gamma \phi x \left[\frac{1}{\gamma}\left(\frac{k}{\phi} n - 1\right) - x \right] \label{logistic}
\end{equation}
に書き換えられる.

ここでは,代謝物Xとして,細胞を構成する分子(たとえば細胞膜を作るリン脂質や細胞骨格など)を想定している.
そして,ここで言う「消費」とは,細胞質を拡散している分子が細胞の膜や骨格として固定されることを指している.
しかしその一方で,代謝物Xを,ある速度で分解するリボソームと捉えても良い.
その場合,分解速度は$\phi$であり,成長速度$\mu$はリボソーム濃度$x$に比例する(比例係数$\gamma\phi$),と解釈することができる.
また,このような現実の細胞との対応という観点で言えば,代謝物Xを生成する自己触媒反応は,実際の代謝反応をかなり簡単に捉えたものにすぎない.

\subsection{結果と考察}
方程式\eqref{logistic}の定常解のうち安定なものは,栄養濃度$n$によって
\begin{equation}
  \begin{cases}
    \frac{k}{\phi} n - 1 > 0 \text{のとき} & x = \frac{1}{\gamma} \left( \frac{k}{\phi} n - 1 \right) \\
    \frac{k}{\phi} n - 1 < 0 \text{のとき} & x = 0
  \end{cases}
\end{equation}
と分岐(トランスクリティカル分岐)する.
したがって,成長速度が正,つまり$x > 0$であるには,栄養濃度$n$は$n > n^* = \phi/k$を満たす必要がある.
つまり,細胞が成長できる栄養濃度には下限が存在し,それは代謝物の消費反応の速度定数を生成反応の速度定数で割った値に等しい.

また,一般に成長速度$\mu$が代謝物濃度$x$の単調増加関数$\mu(x)$である場合にも,$\mu(0)=0$(代謝物が存在しないとき細胞は成長しないという条件)が成り立つならば,まったく同じことがいえる.
実際,この場合には方程式\eqref{logistic}のかわりに
\begin{equation}
  \frac{dx}{dt} =  x \left(kn - \phi - \mu(x)\right) 
\end{equation}
が成り立ち,安定な定常解は
\begin{equation}
  \begin{cases}
    \frac{k}{\phi} n - 1 > 0 \text{のとき} & x = \mu^{-1}\left(kn - \phi \right) >0 \\
    \frac{k}{\phi} n - 1 < 0 \text{のとき} & x = 0
  \end{cases}
\end{equation}
となる.ただし,$\mu(x)$の逆関数を$\mu^{-1}(x)$とした.
