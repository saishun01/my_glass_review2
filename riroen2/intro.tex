細胞内の代謝反応は様々な形でモデル化され,理論的に研究されてきた.
たとえば,非常に多数の成分からなる自己触媒反応ネットワークを考えることで,細胞内の分子数がZipfの法則に従うことが明らかになった\cite{hurusawa}.
ほかにも,代謝がある確率で失敗することでゴミ分子が生成され,それが正常な代謝を阻害するようなモデルも考えられた\cite{hk17}.
このモデルは,低栄養環境に置かれた細胞が高栄養環境に移されたときに成長を再開するのにかかる時間(ラグタイム)をよく説明することが分かった.
こうしたモデルは分子の詳細によらないため,様々な生体分子に関して普遍的に成り立つ法則を含んでいるといえる.

こうした研究を受け,私は,細胞が成長できる条件を分子の詳細によらない形で記述することを考えた.
具体的には,ごく簡単な自己触媒反応のモデルにより,栄養濃度がある閾値を下回ると細胞が成長できないことを明確に説明し,その閾値を解析的に導出した.
次に,このモデルの化学反応を若干変更した2つのモデルについて栄養濃度の閾値を再び求め,もとのモデルと比較して考察した.

なお,本レポートの内容の多くは解析的な議論であり,それは理論演習の授業がすべて終了した後に個人的に行ったものである.
そのため,授業で発表した数値計算の結果に対する考察は少なくなってしまった.
しかし,ここで扱っている細胞モデルは,基本的に授業で発表したモデルのいずれかと等価なので,授業と無関係ではないことをくみ取っていただければ幸いである.
