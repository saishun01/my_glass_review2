\section{ヘモグロビンの役割}
これまでヘモグロビンと酸素の結合について数理的に考えてきた.
しかし,そもそもなぜヘモグロビンはミオグロビンとは違ってここまで複雑なことをやっているのだろうか.
その理由はヘモグロビンの働きにある.
ヘモグロビンは肺で酸素を効率よく受け取るために,肺ではできるだけ多くのサイトを酸素で埋めた方が良い.
一方で肺以外の部位では,酸素を効率よく手放して各組織に受け渡すために,占有率は極力低い方が良い.
アロステリック効果による占有率のS字カーブは,この両方を見事に実現している.

一方でミオグロビンは,ヘモグロビンから受け取った酸素を蓄える働きをするので,ヘモグロビンのサイトの占有率が低いときであっても高い占有率が求められる.
これは単純な反応による双曲線のカーブによって実現される.

また,ヘモグロビンに働くアロステリック効果は他にもある.
たとえばpHが低い状況では,ヘモグロビンが水素イオンと結合することで酸素との結合が起こりづらくなることが知られている(Bohr効果)\cite{bohr}.
この効果により,血中の二酸化炭素濃度が高いと,二酸化炭素が水と反応して水素イオンを放出する(pHを低くする)ため,ヘモグロビンは酸素を離しやすくなる.
一般に,肺では酸素濃度が高く二酸化炭素濃度が低いのに対し,組織では酸素濃度が低く二酸化炭素濃度が高い.
そのため,Bohr効果によってヘモグロビンは肺で酸素を結合して組織でそれを手放すという仕事をさらに効率的に行うことができる\footnote{この話は高校の生物(生物基礎)の教科書に載っている.}.

このように,生物は必要に応じてアロステリック効果を利用し,生命活動を適切に制御している.
その意味で,アロステリック効果はアロステリック制御とも呼ばれる.