\section{MWCモデル}

MWCモデルでは,受容体が$n$個のサブユニットからなり,それぞれのサブユニットに一つずつリガンド結合サイトがあると考える.(ヘモグロビンと酸素の結合を考える際は$n=4$である.)
さらに,受容体が二つの状態(tense状態とrelaxed状態.それぞれT,Rと表す)の間を遷移すると考える.
T状態の受容体に$i$個のリガンドが結合した状態を$\ce{T_{$i$}}$,R状態の受容体に$i$個のリガンドが結合した状態を$\ce{R_{$i$}}$と書く.
また,T(R)状態の受容体の空きサイトにリガンドが結合する反応の速度定数は$k_{\mathrm{on},\ce{T}}$($k_{\mathrm{on},\ce{R}}$),その逆反応は$k_{\mathrm{off},\ce{T}}$($k_{\mathrm{off},\ce{R}}$)で一定とする.
このとき,各反応の会合定数(解離定数の逆数)は
\begin{equation}
  K_{\ce{T}} = \frac{k_{\mathrm{off},\ce{T}}}{k_{\mathrm{on},\ce{T}}} \qquad K_{\ce{R}} = \frac{k_{\mathrm{off},\ce{R}}}{k_{\mathrm{on},\ce{R}}}
\end{equation}
となる.

また,$n$個のサブユニットを区別し,どのような順番でもリガンドが結合できると仮定する.
たとえばリガンドが1つだけ結合した$\ce{T1}$は,実際には「どのサブユニットにリガンドが結合しているか」によって区別された$n$通りの状態を含むと考える.
それを加味してレート方程式を考えると,以下が成り立つ:
\begin{align}
  \dv{t}[\ce{T0}] &= k_{\mathrm{off}, \ce{T}}[\ce{T1}] - n k_{\mathrm{on}, \ce{T}}[\ce{T0}] + J_{\ce{RT},0} \\
  \dv{t}[\ce{T_$i$}] &= k_{\mathrm{off}, \ce{T}}[\ce{T_$i$}] - n k_{\mathrm{on}, \ce{T}}[\ce{T_$i-1$}] + J_{\ce{RT},i} 
\end{align}
ただし,$J_{\ce{RT},i}$は$\ce{R_$i$}$と$\ce{T_$i$}$の間の遷移による$\ce{T0}$の正味の流入率を表す.

最初の仮定として,T状態とR状態の間の化学平衡は素早く達成される(常に平衡にある)と考えて,$J_{\ce{RT},i}=0$とする.



定常状態において$i=1,2,3,4$に対して
\begin{align}
  [\ce{T_{$i$}}] &= _n C _i K_{\ce{T}} [\ce{T_{$i-1$}}] [\ce{O2}] \\
  [\ce{R_{$i$}}] &= _n C _i K_{\ce{R}} [\ce{R_{$i-1$}}] [\ce{O2}] 
\end{align}
が成り立つ.

ここで,受容体のT状態とR状態の間の遷移が定常となる条件を考える.



