\documentclass[12pt, a4j, dvipdfmx, sffamily]{jsarticle}
\usepackage[dvipdfmx]{graphicx}
\usepackage[margin=9truemm, bottom=18truemm]{geometry}
\usepackage{otf}
\special{}

\usepackage{amsmath, physics}
\usepackage{amssymb}
\usepackage{lipsum}



%色
\usepackage{xcolor}
\definecolor{lg}{rgb}{0.56, 0.93, 0.56} %lightgreen
\definecolor{color1}{rgb}{0.918, 0.402, 0.430}%赤
\definecolor{color3}{rgb}{0.484, 0.750, 0.316}%緑

\colorlet{thiscolor}{color3}	%テーマカラーの選択
\colorlet{linecolor}{thiscolor!150}
\colorlet{pagecolor}{thiscolor!40}
\pagecolor{pagecolor}


%ref
\usepackage[dvipdfmx, colorlinks = true, linkcolor = lg, urlcolor=blue, citecolor = lg, anchorcolor = lg]{hyperref}
\usepackage{pxjahyper, url}


%ヘッダーとフッター
\usepackage{fancyhdr}
\pagestyle{fancy}
\fancyhead{}
\fancyfoot[C]{\gtfamily\sffamily\textcolor{gray}{生物物理班\hfill\thepage}}
\renewcommand{\headrulewidth}{0pt}


%図
\usepackage{caption}
\usepackage{here}
\usepackage{wrapfig}
\captionsetup[figure]{format=plain, labelformat=simple, labelsep=period, font={sf, footnotesize}}
\renewcommand{\figurename}{Figure}
\renewcommand{\tablename}{Table}


%TikZ
\usepackage{tikz}
\usetikzlibrary{shapes.geometric}


%section関係
\usepackage{titlesec}
\titleformat{name=\section, numberless}
   [hang]
   {\vspace{2pt}\fontsize{20pt}{20pt}\selectfont}
   {\textcolor{linecolor}{■}}
   {3pt}
   {\hspace{0pt}}
   [\vspace{-5mm}\textcolor{linecolor}{\hrulefill}\vspace{-5mm}]


%tcolorbox
\usepackage{tcolorbox}
\tcbuselibrary{raster, skins}
\tcbset{	enhanced,
		colback=white,
		colframe=thiscolor!200,
		sharp corners,
		rounded corners = southeast,
		drop shadow,
		fonttitle=\gtfamily\sffamily\Large,
		fontupper=\gtfamily\sffamily,
		before upper = \parindent 1zw,
		raster column skip = 5mm,
		raster row skip = 5mm,
		raster width center=\linewidth,
		raster valign=top}

%multicol
\usepackage{multicol}
\usepackage{comment}
\setlength{\columnsep}{5mm}
\newenvironment{twocols}{\begin{multicols}{2}[\vspace{-4mm}]}{\end{multicols}}
\newenvironment{twocols*}{\begin{multicols*}{2}[\vspace{-4mm}]}{\end{multicols*}}
%%%%%%%%%%%%%%%%%%%%%%%%%%%%%%%%%%%%%%%%%%%%%%%%
\begin{document}\sffamily\gtfamily
\newpage
\section*{ポスターテンプレート}
\begin{tcbraster}[raster columns = 1]
	\begin{tcolorbox}[title = 本文・\texttt{\textbackslash tcbline}]
	これは本文です.数式を入れることもできます:
		\begin{equation}
		\dv[2]{x}{t}
		=-ax-b\dv{x}{t}.
		\end{equation}
	\tcbline
	必要に応じて\texttt{\textbackslash tcbline}で区切りを入れます.
	\end{tcolorbox}
%%%
	\begin{tcolorbox}[title = 高さ , height = 30mm]
	高さを手動で指定する場合には\texttt{tcolorbox}環境のオプションで\texttt{height}を指定すれば良いです.
	\end{tcolorbox}
%%%
	\begin{tcolorbox}[title = タイトルの指定]
	各\texttt{tcolorbox}には見出しをつけることを推奨します.\texttt{tcolorbox}環境の\texttt{title}オプションでタイトルを設定できます.
	\end{tcolorbox}
%%%
	\begin{tcolorbox}[title = ポスター見出し]
	ポスターの見出しには\texttt{\textbackslash section*}を使ってください.
	\end{tcolorbox}
%%%
	\begin{tcolorbox}[title = lipsum1]
	以下のテスト用の文章で,\texttt{lipsum}パッケージを使っています.
	\tcbline
	\lipsum[1]
	\end{tcolorbox}
\end{tcbraster}
%%%%%%%%%%%%%%%%%%%%%%%%%%%%%%%%%%%%%%%%%%%%%%%%
\newpage
\section*{ポスターテンプレート2}
\begin{tcbraster}[raster columns = 1, raster equal height]
%%%
	\begin{tcolorbox}[title = \texttt{raster equal height}]
	\texttt{tcbraster}環境のオプションで\texttt{raster equal height}をつけると,全ての\texttt{tcolorbox}(枠のこと)の高さを等しくできます.
	
	タイプセットを2回する必要があるので注意してください.
	\end{tcolorbox}
%%%
	\begin{tcolorbox}[title = lipsum2]
	\lipsum[2]
	\end{tcolorbox}
%%%
	\begin{tcolorbox}[title = lipsum3]
	\lipsum[3]
	\end{tcolorbox}
\end{tcbraster}

%%%%%%%%%%%%%%%%%%%%%%%%%%%%%%%%%%%%%%%%%%%%%%%%
\newpage
\section*{ポスターテンプレートポスターテンプレートポスターテンプレートポスターテンプレート3}
\begin{twocols}
\begin{tcbraster}[raster columns = 1]
	\begin{tcolorbox}[title = {\texttt{twocols}環境, \texttt{\textbackslash columnbreak}}]
	2段組の際には\texttt{twocols}環境を使ってください.スペースを既に調整してあります.
	(\texttt{tcbraster}環境を使った方法は後述.)
	
	段組を改める場合には\texttt{\textbackslash columnbreak}とします.
	\tcbline
	\texttt{twocols}環境は\texttt{multicols}環境のスペースを少しだけいじったものです.
	\tcbline
	特に自動で段組をさせたくないときには\texttt{twocols*}環境を使ってください.
	ページの最後までが2段組になります.
	\tcbline
	\lipsum[1][1-3]
	\end{tcolorbox}
\columnbreak
	\begin{tcolorbox}[title = lipsum4]
	\lipsum[4]
	\end{tcolorbox}
%%%
	\begin{tcolorbox}[title = lipsum5]
	\lipsum[5]
	\end{tcolorbox}
\end{tcbraster}
\end{twocols}

%%%%%%%%%%%%%%%%%%%%%%%%%%%%%%%%%%%%%%%%%%%%%%%%
\newpage
\section*{ポスター\\テンプレート4}
\begin{tcbraster}[raster columns = 2, raster equal height = rows]
	\begin{tcolorbox}[title = 1. 横のものと高さを揃える]
	横の\texttt{tcolorbox}と高さを揃えたい場合には\texttt{tcbraster}環境のオプションで\texttt{raster equal height = rows}とします.
	\end{tcolorbox}
%%%
	\begin{tcolorbox}[title = 2. lipsum6]
	\lipsum[6]
	\end{tcolorbox}
%%%	
	\begin{tcolorbox}[title = 3. 順番]
	\texttt{tcbraster}環境のオプション\texttt{raster coumns = 2}で2段組にした場合, 左上・右上・左下・右下の順番に表示されることに注意してください.
	
	縦に順番に表示する\texttt{twocols}環境とは異なります.
	\tcbline
	したがって,時系列が重要でない「箇条書きのような」使い方に適しています.
	\tcbline
	もし,\texttt{twocols}環境のような順番で表示したい場合には1, 3, 2, 4の順番にしておけば良いです.
	\end{tcolorbox}
%%%	
	\begin{tcolorbox}[title = 4. lipsum7-1]
	\lipsum[7][1]
	\end{tcolorbox}
%%%
\end{tcbraster}

%%%%%%%%%%%%%%%%%%%%%%%%%%%%%%%%%%%%%%%%%%%%%%%%
\newpage
\section*{ポスターテンプレート5}
\begin{tcbraster}[raster columns = 2, raster equal height = rows]
	\begin{tcolorbox}[title = \texttt{raster multicolumn}, raster multicolumn = 2]
	ページの中で1組と2段を両方使う場合には,\texttt{tcbraster}環境のオプションを\texttt{raster columns = 2}として全体を2段組とした上で,1段組にしたい\texttt{tcolorbox}にオプション\texttt{raster multicolumn = 2}とつけます.
	\end{tcolorbox}
%%%
	\begin{tcolorbox}[title = lipsum7-2]
	\lipsum[7][2-17]
	\end{tcolorbox}
%%%
	\begin{tcolorbox}[title = lipsum8]
	\lipsum[8]
	\end{tcolorbox}
%%%
	\begin{tcolorbox}[title = lipsum9, raster multicolumn = 2]
	\lipsum[9]
	\end{tcolorbox}
\end{tcbraster}

%%%%%%%%%%%%%%%%%%%%%%%%%%%%%%%%%%%%%%%%%%%%%%%%
\newpage
\section*{ポスターテンプレート6}
\begin{tcbraster}[raster columns = 1]
%%%
	\begin{tcolorbox}[title = 画像(\texttt{figure}環境)]
	\texttt{figure}環境は使えますが,\texttt{h},\texttt{t},\texttt{b}などのオプションは使えません.
	\texttt{tcolorbox}環境内であることが理由です.
	
	代わりに,\texttt{here}パッケージの\texttt{H}オプション等を用いるなどの対処をしてください.
	\begin{figure}[H]
	\centering
	\includegraphics[width=0.2\linewidth]{logo_test.png}
	\caption{Physics Lab.~2021のロゴ.}
	\end{figure}
	\end{tcolorbox}
%%%	
	\begin{tcolorbox}[title = 画像(\texttt{wrapfigure}環境), height = 0.4\linewidth]
	\begin{wrapfigure}{r}[0pt]{0.2\linewidth}
	\centering
	\includegraphics[width=\linewidth]{logo_test.png}
	\caption{Physics Lab.~2021のロゴ.}
	\end{wrapfigure}
	%%
	\texttt{wrapfigure}環境を使って文字の回り込みをできます.
	ポスターはスペースの制限が大きいので,活用してください.
	
	\texttt{tcolorbox}の高さは\texttt{wrapfigure}環境の高さに影響されないようなので,\textgt{手動で}指定してください.abcd\textbf{abcd}
	\end{tcolorbox}
%%%	
\end{tcbraster}

\begin{figure}[H]
\centering
\includegraphics[width=0.7\linewidth, trim = 50 50 50 50, clip]{test.eps}
\end{figure}




\end{document}