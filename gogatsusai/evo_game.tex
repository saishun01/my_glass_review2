\section{ゲーム理論で見る進化}
こうしたアプローチが適用できるのは,適応度が「相手(同じ集団に属する他の生物)によらない」場合である.
しかし必ずしもそうとは限らない.
もし生物個体の適応度が他の生物によって変わる場合,そこには他の生物との間の駆け引きが生まれる.
このような駆け引きは,経済学に登場するゲーム理論を用いることで数理的に扱うことができる.
以降ではゲーム理論的な考えを使った考察を紹介する.

\subsection{進化ゲームの設定}
最初に,参考文献\cite{game}に基づいて,生物進化がゲーム理論における戦略形ゲームと対応づけて記述できることを見る.
ここで扱うゲームは「進化ゲーム」と呼ばれる.

まず$n$体の生物からなる集団を考え,生物個体に$i=1,2,\cdots ,n$の番号をふって区別する.
個体はいくつかの選択肢を持ち,それらを確率的に選択すると考える.
たとえば$s$,$t$という2通りの選択肢があるとき,個体は確率$p$で$s$をとり,確率$1-p$で$t$をとるという行動が許される.
このように生物個体の選択肢をとる確率が分かれば,その個体の振る舞い(戦略)が決まる.
とくに,どれか一つの選択肢だけを(確率1で)選ぶような戦略を純粋戦略といい,複数の選択肢を確率的に組み合わせた戦略を混合戦略という.
ここでは選択肢の集合は全個体に共通と考え,それを$S$とおく\footnote{つまり個体ごとの個性は選択肢ではなくその選び方(確率分布)に現れると考える.生物集団が同種の個体からなるとき,選択肢の集合$S$が種の特徴を表していると考えれば,これは自然な仮定である.}.
そして個体$i$が選択肢$s\in S$を選ぶ確率を$p_i(s)$とする.
このとき,確率分布$p_i$は個体$i$の(混合)戦略と一対一対応し,以降ではこれを``戦略$p_i$"と呼ぶことにする.

また,生物個体がとる戦略はその個体の遺伝子で決定され,途中で変えることはできないと考える.
そういう意味で,戦略とは生物個体の生まれつきの性質であるといえる.

次に,いよいよ個体間の競争を考える.
まず,生物集団内の異なる二個体$i,j$が出会うと,それらは各々の戦略に従ってそれぞれ選択肢を選ぶ.
まず簡単に,個体$i$は選択肢$s\in S$を,個体$j$は選択肢$t\in S$を確率1で繰り出す純粋戦略をとるとする.
そしてこの対戦の結果,個体$i$の利得$f_i(s,t)$と個体$j$の利得$f_j(s,t)$が決まるとする.
ここで,集団内の個体はみな対等なので
\begin{equation}
  f_i(s,t) = f_j(t,s) \eqqcolon f(s,t)
\end{equation}
が成り立つ\footnote{このようなゲームを対称であるという.}.
簡単に言えば,$f(s,t)$とは$s$と$t$が対戦したときの$s$側の利得である.
次に,個体$i,j$がそれぞれ戦略$p,q$をとる場合を考える.
このとき,この駆け引きで見込める個体$i$の期待利得(利得の期待値)は,
\begin{equation}
  F(p,q) \coloneqq \sum_{s,t\in S} p(s) q(t) f(s,t) \label{F}
\end{equation}
と書ける.
$F$は確率分布$p,q$に関して双線形な(汎)関数である.
以降では,戦略同士の対戦結果として式\eqref{F}をそのまま採用する.

ここでは,このような対戦がどの個体間でも均等に十分な回数行われると考える.
このとき個体$i=1,\cdots, n$の期待利得はその期待値
\begin{equation}
  F_i \coloneqq \frac{1}{n-1} \sum_{j=1, j\neq i}^{n} F(p_i, p_j) \label{fitness}
\end{equation}
に収束すると考えられる.
ただし,個体$k=1,\cdots,n$は戦略$p_k$をとるとした.
次世代では戦略$p_i$をとる個体の数が$F_i$倍になると考える\footnote{$F_i$倍の結果が整数とは限らないので,あくまでこれは目安である.もし次世代についても同様の議論をしたければ,各戦略$p_i$をとる個体数は$F_i$倍の結果に最も近い整数で近似するべきと考える.}.
つまり,このゲームにおける利得とは次世代にどれだけ自分と同じ戦略を残せるかに対応する.
この値$F_i$を改めて個体$i$の適応度と呼ぶ.

\subsection{進化的安定戦略}
以上の設定をもとに,適応ダイナミクスと呼ばれる過程を考える.
まず,すべての個体が同じ戦略$p$をとる生物集団を考える.
この集団の中で突然変異が起こり,戦略$q$をとる個体が現れたとする.
具体的には,戦略$p$をとる個体(野生型)と戦略$q$をとる個体(変異型)の頻度が$(1-\epsilon) : \epsilon$であると考える($\epsilon$は微小な正数とする).
つまり,集団に属する個体数を$n$として,野生型の個体数は$n(1-\epsilon)$,変異型の個体数は$n\epsilon$である.
このとき式\eqref{fitness}より,野生型の適応度$F_p$は
\begin{equation}
  F_{p} = \frac{1}{n-1} \qty[\qty{n(1-\epsilon)-1} F(p,p) + n\epsilon F(p,q)] 
\end{equation}
であり,変異型の適応度$F_q$は
\begin{equation}
  F_{q} = \frac{1}{n-1} \qty{n(1-\epsilon) F(q,p) + (n\epsilon-1) F(q,q)}
\end{equation}
である.
よって,個体数$n$が十分大きい場合を考えることにすると
\begin{equation}
  F_{p} \approx (1-\epsilon)F(p,p) + \epsilon F(p,q) ,\qquad F_{q} \approx (1-\epsilon)F(q,p) + \epsilon F(q,q) \label{fpfq}
\end{equation}
となる.
したがって,もし$F_p < F_q$なら次世代には変異型の割合が増えることになる.
一方でもし$F_p > F_q$なら次世代には変異型の割合が減ることになる.
後者の場合,野生型の戦略$p$が変異型の戦略$q$に打ち勝って,集団全体で戦略$p$が主流であり続けることになる.
このとき,戦略$p$は進化的安定戦略(ESS)であるという\footnote{ESSの存在しないゲームもある(参考文献\cite{game}).}.
ESSの厳密な定義は次のように与えられる.

\begin{dfn}[進化的安定戦略]
  \label{ess}
  戦略$p$が進化的安定戦略(evolutionarily stable strategy:ESS)であるとは,$p$とは異なる任意の戦略$q$に対してある$\epsilon_0 >0$が存在して,$0<\epsilon\le \epsilon_0$となる任意の$\epsilon$に対して,
  \begin{equation}
    (1-\epsilon)F(p,p) + \epsilon F(p,q) > (1-\epsilon)F(q,p) + \epsilon F(q,q)
  \end{equation}
  が成り立つことである.
\end{dfn}

つまり,$\epsilon$が十分小さいときに任意の変異型に打ち勝てるなら,その戦略はESSである.
しかし,実際にある戦略がESSか判定するときに毎回$\epsilon$を考えるのは面倒である.
そこで次の定理がよく用いられる.
\begin{thm}
  戦略$p$がESSであるための必要十分条件は,次の2つの性質が成り立つことである.
  \begin{enumerate}
    \item すべての戦略$q$に対して,$F(p,p) \ge F(q,p)$.
    \item $F(p,p) = F(q,p)$となる任意の戦略$q(\neq p)$に対して,$F(p,q) > F(q,q)$.
  \end{enumerate}
\end{thm}

これは関数$F$の双線形性と連続性から示せる.
証明は参考資料\cite{game}にある.