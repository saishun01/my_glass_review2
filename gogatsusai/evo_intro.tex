\section{はじめに}

生物は非常によくできている.
遺伝子発現の制御や複雑な代謝経路,免疫系など,生物の教科書を少し眺めるだけでもその緻密さを実感できると思う.
では,なぜこのような仕組みが現在まで保たれてきたのだろうか.
また,現在地球上で様々な生物が見られるが,なぜ生物はそこまで多様になったのだろうか.
こうした生物に対する「なぜ」という問いに答える上で,生物進化の観点は欠かせない.
ここではこの生物進化のプロセスを説明し,さらにそれを数理的な観点で捉える試みを紹介しようと思う.

\section{生物進化の基本}
はじめに,参考文献\cite{text}に基づいて,生物の進化がどのように起こるかを(数式を使わずに)言葉で説明する.

\subsection{遺伝子の発現}
まず,生物はみなデオキシリボ核酸(DNA)という物質を持っている.
DNAはヌクレオチドと呼ばれる単位が鎖状に連なった高分子である.
ヌクレオチドは糖・リン酸・塩基の三つからなり,塩基にはアデニン(A),グアニン(G),シトシン(C),チミン(T)の4種類がある.
DNAはこの4種類の塩基の並び(塩基配列)を情報として持つ.
また,DNAは細胞分裂のタイミングに合わせて複製されるようになっている.

そしてこのDNAは転写と翻訳というステップを経て,最終的に塩基配列に応じて特定の種類のタンパク質が作られる\footnote{転写と翻訳についての詳しい話は生物の教科書に譲る.}.
タンパク質は生物体を構成し,生物の代謝や物質の輸送にも関わる.
ただし,DNAの塩基配列のうちタンパク質の設計に携わる部分は限られており,それを遺伝子と呼ぶ.
つまり,この遺伝子が生物の性質を生み出すタンパク質のもとになっている.
遺伝子からタンパク質へ至るこのような流れのことを遺伝子の発現と呼ぶ.

\subsection{突然変異と自然選択}
遺伝子の発現の話から分かるように,生物個体の性質は遺伝子に由来する.
さらに遺伝子は親から子へと受け継がれ,それによって親の形質は子の形質に受け継がれる\footnote{具体的な受け継がれ方はここでは深入りしない.}.
「蛙の子は蛙」というのは,まさしく親蛙の遺伝子が子蛙に継承されるからである.
しかし,たとえばDNAの複製の過程でミスが起こると,親と違う遺伝子が子に現れることがある\footnote{複製ミスのほかに,染色体(真核生物に見られる,DNAがヒストンと呼ばれるタンパク質に巻き付いた構造のこと)の構造変化なども要因として挙げられる.}.
このような遺伝情報の変化を突然変異といい,生まれた新しい方を変異型,もとの方を野生型という.

突然変異は生物の生存にほとんど影響しないこともあれば,決定的に関わることもある.
後者の場合,たとえば変異型が野生型の持つ重要な機能を損なっていれば,変異型は死んでしまうかほとんど子孫を残せず,その遺伝子は集団から消えていく.
一方で変異型が野生型よりも生存に有利な性質を持っていれば,むしろ野生型よりも子孫を残し,その遺伝子は集団全体に広がっていく.
このような過程を自然選択といい,「次世代に残せる遺伝子のコピー数」を適応度という.
まとめると,自然選択によって適応度の高い(子孫を多く残せる)個体のもつ遺伝子が集団全体に広がる.
このような仕組みで,世代が進むごとに集団全体を通して適応度は高くなっていく.


\section{最適化で見る進化}
以上の仕組みを考えれば,生物の集団は適応度を最大化する方向に進化していくことが分かる.
逆に言えば,現在の生物の振る舞いを「適応度の最大化」問題として導くことができると考えられる.
たとえば,植物の体には光合成を行う栄養器官(葉)と繁殖器官(花)があり,植物は成長過程でその都度どちらにどれだけ栄養を割り当てるか(どれだけ大きくするか)を決めている.
栄養器官に栄養を回せばその後に使える栄養の量を増やせるが,それだけでは繁殖器官が発達しないので子孫を残せない.
結局,適応度が最も高いのは植物の最期において繁殖器官を最大にするような割り当てと考えられる.
そこで実際にモデルを立ててそのような最適な割り当てを計算し,実際の植物の振る舞いと比較するという研究がある(詳細は参考文献\cite{text}にある.).