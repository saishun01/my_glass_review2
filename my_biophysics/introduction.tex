\section{はじめに}
現在,「生命とは何か」という問いが真剣に検討されている.

この問いはまず,文字通りに生命の定義を問うていると捉えることができる.
これまでの研究の中で生命には様々な定義がなされてきた.
たとえば生命科学の教科書では,生命の定義として,(1)細胞を持ち,(2)自己増殖し,(3)代謝を行い,(4)恒常性を有し,(5)親から子へと遺伝が起こる,といったことを挙げている.
しかし,これらの多くは生命の定義というよりも現存する地球上の生物が概ね共通して持つと考えられる特徴の列挙にすぎない.
またこのような定義を多少書き換えて,ウイルスを生命に含まないようにすることもあるし,含むようにすることもある.
一方で,地球外生命体の探査に乗り出したNASAでは,生命を「ダーウィン進化を行いうる,自己維持できる化学システム」と定義している.
この一般化された生命の定義からは,地球上の生命とはまるで違っている地球外の「何か」をも生命に含めようという意思を感じる.
このような点で,生命の定義は究極的には,我々が生命をどう定義したいか,あるいはどう定義するのが自然に感じられるか,という問題に帰着する.

一方で,「生命とは何か」という問いは,生命の仕組みを問うているとも捉えられる.
もし生命の本質と思えるような仕組みがあれば,それを生命の定義にできるからである.
本稿ではこの立場に立って,生命の仕組みを極力一般的な立場から俯瞰する.
ただし,未だ見つかっていない地球外生命体には言及せず,地球上の生命について論じる.
また,必要に応じて地球上の生物の具体例や数値を持ち出すことで,一般論の妥当性を保証する.
もちろん,例外を恐れていては深い議論ができないので,ある程度普遍的に見られる機構にも踏み込んで説明する.
