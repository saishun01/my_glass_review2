\section{進化}

生命には基本的に親が存在する.
親に対して,その生命を子と呼ぶ.
生命の場合,この親と子の間には密接な関係がある.
具体的には,子は親から遺伝子を継承する.
遺伝子はDNA上に存在し,転写・翻訳を通してタンパク質として発現する.
タンパク質は生命の代謝や輸送などの役割を担っている.

しかし,遺伝子が正しく継承されない場合もある.
つまり、子の遺伝子の中に親のものから変異することがある.
ある個体の遺伝子に変異が起こると,形質転換が起こり,その結果,他の個体と比べてその個体が残せる子の数が変わることがある.
こうして自然選択が起こる.