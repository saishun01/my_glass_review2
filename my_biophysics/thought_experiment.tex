\section{思考実験}
ここでは,生命の本質を理解するためにいくつかの思考実験を行う.
これらの思考実験は,一度生命の複雑な機構を忘れさせ,それが生まれる以前,つまり生命の起源を考えさせるものである.
ここで考えたいのは,何が生命を生命たらしめているのかということである.
これを探るために,現存する生命の持つ重要な特徴をそぎ落とした仮想的な生命を考える.
しかし,その仮想的な生命を極力生命らしくするように努める.
それを通して,この仮想的生命は果たして生命と呼べるかを考えていきたい.

\subsection{複製しない生命}
まず,生命から複製能を取り除こう.
そうすると,その生命は細胞を分裂させることができない.
つまり,自分のクローンも子供もつくることができないため,遺伝どころか親や子といった概念すら失うことになる.
つまり,この生命は一世代きりのものであり,自己維持に失敗すればそれで絶滅する.

しかし,まだ諦めるには早い.
絶滅後には,再び同じような生命が誕生するまで空白の期間があり,そしてまた一世代で絶滅し,というサイクルが繰り返されるとしよう.
こうすれば断絶した一世代同士を結ぶことで,「世代」の概念が新しい形で復活する.
これを改めてこの生命の「世代」と考えれば,もはやこの生命は一世代きりのものではなくなる.

この段階では,まだこの生命は周期的に組織化とその崩壊を繰り返すだけである.
そこで,進化能を何とか獲得させてみよう.
たとえば,絶滅後に崩壊した生命の断片が配置を変え,次の世代の生命を構築すると考えよう.
このとき,断片に次世代の配置を制限する情報を蓄積できたとする.
たとえば,その生命の寿命が短いほど強く「この配置を避けよ」という情報を断片に刻み込めば,次世代はこれとは違う配置になる.
それによって寿命が短くなればまた次の世代でそれを避け,寿命が長くなれば同じような変化を次の世代に与えていく.
つまり,断片としてメモリを考え,そこに記憶を蓄積するのである.
そのプロセスの収束先は,そこそこの寿命を持った生命となるはずであり,これはある種の(本来の意味とは異なるが)進化といえる.

しかし現実的には,断片に刻める情報にも限界があるだろう.
そのため,どこかで配置が元に戻り(進化がリセットし),再び同じことの繰り返しになる可能性がある.
その場合,結局長期的に見れば進化能は失なわれてしまう.

また,このようなプロセスは自然界ではほとんど起こらないと言って良い.
断片はメモリである以前に物質なので,ある確率で分解し,情報はどこかに消えてしまう.
さらに死んだ生命が崩壊する際に断片が拡散したとすれば,断片はどんどん散らばっていき,再び凝集して生命を形成する確率は極めて小さい.
そのため,結局次の世代が誕生することはないだろう.

まとめると,生命は複製能を失ったとしても世代や進化を定義し直すことは不可能ではない.
しかしながら,有限の物質に刻み込める情報が有限であることを考えれば,進化は不完全である.
また,物質の分解や拡散を考えれば,結局一世代きりになりそうである.

個人的には,これは生命とは呼べないと考える.
というのも,世代や進化はうまくいっていない以上,これは単なる「一時的に凝集して再び自然に返った」だけの存在だからである.
生命と言うからには,数世代にわたって持続可能な,ロバストな機構が必要だと考える.

\subsection{代謝しない生物}
生命から代謝を取り除こう.
その結果,