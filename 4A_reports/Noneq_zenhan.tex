\documentclass[a4paper,11pt]{jsarticle}


% 数式
\usepackage{amsmath,amsfonts}
\usepackage{amssymb}
\usepackage{bm}
\usepackage{physics}
% 画像
\usepackage[dvipdfmx]{graphicx}
% ローマ数字
\usepackage{otf}
% 単位
\usepackage{siunitx}
% 表
\usepackage{multirow}
% 化学反応
\usepackage[version=4]{mhchem}


\begin{document}

\title{非平衡科学 前半レポート}
\author{05-211525 齋藤駿一}
\date{\today}
\maketitle

\section{}
以下の2次元Fokker-Planck方程式(引数は省略した)を考える.
\begin{align}
  \partial_t P_{X,Y} &= -\partial_x \left(\nu_X P_{X,Y}\right) - \partial_y \left(\nu_Y P_{X,Y}\right), \label{Pt}\\
  \nu_X &= \mu F_X - \mu \beta^{-1} \partial_x \ln{P_{X,Y}}, \label{nX}\\
  \nu_Y &= \mu F_Y - \mu \beta^{-1} \partial_y \ln{P_{X,Y}}. \label{nY}
\end{align}

\subsection*{1.1}
平衡とは,任意の場所で確率の流れがゼロとなる定常状態のことである.
つまり,任意の$(x,y)$について
\begin{equation}
  \nu_X(x,y;t) = \nu_Y(x,y;t) = 0 \label{eq}
\end{equation}
が成り立つことである.

\subsection*{1.2}
定常とは,任意の場所で確率分布が時刻によらない状態のことである.
つまり,任意の$(x,y)$について
\begin{equation}
  \partial_t P_{X,Y}(x,y;t) = 0 \label{st}
\end{equation}
が成り立つことである.

\subsection*{1.3}
講義資料によると,Fokker-Planck方程式
\begin{equation}
  \partial_t P_X(x;t) = - \partial_x [A^{(1)}(x;t)P_X(x;t)] + \frac{1}{2} \partial_x^2 [A^{(2)}(x;t)P_X(x;t)]
\end{equation}
にはLangevin方程式
\begin{equation}
  \dv{x}{t} = A^{(1)}(x(t);t) + \sqrt{A^{(2)}(x(t);t)}\cdot \xi(t)
\end{equation}
が対応する.
今回の系はx成分とy成分を独立に扱えるので,それぞれについて上のような対応を考えると,x成分については
\begin{equation}
  A^{(1)} = \mu F_X, \qquad A^{(2)} = 2\mu\beta^{-1},
\end{equation}
y成分については
\begin{equation}
  A^{(1)} = \mu F_Y, \qquad A^{(2)} = 2\mu\beta^{-1}
\end{equation}
となる.
したがって,Fokker-Planck方程式\eqref{Pt}に対応するLangevin方程式は
\begin{align}
  \dv{x}{t} &= \mu F_X + \sqrt{2\mu \beta^{-1}} \cdot \xi(t) \label{dxdt}\\
  \dv{y}{t} &= \mu F_Y + \sqrt{2\mu \beta^{-1}} \cdot \xi(t) \label{dydt}
\end{align}
と与えられる.

\subsection*{1.4}
式\eqref{Pt}に式\eqref{nX}および式\eqref{nY}を代入すると,
\begin{equation}
  \partial_t P_{X,Y} = - \mu \partial_x (F_X P_{X,Y}) + \mu \beta^{-1} \partial_x^2 P_{X,Y}  - \mu \partial_y (F_Y P_{X,Y})+ \mu \beta^{-1} \partial_y^2 P_{X,Y}  
\end{equation}
が得られる.
この右辺の第1項はx方向に働く力$F_X$による駆動を表し,第2項はx方向の拡散を表す.
これらはそれぞれ式\eqref{dxdt}の右辺第1項,第2項に対応する.
同様に,上式の第3項と第4項は,y方向の駆動と拡散なので,それぞれ\eqref{dydt}の右辺第1項,第2項に対応する.

\subsection*{1.5}
力を次のように与える.
\begin{align}
  F_X &= -\partial_x U(x,y), \\
  F_Y &= -\partial_y U(x,y), \\
  U(x,y) &= \frac{k}{2}\left(x^2+y^2\right).
\end{align}
このとき,
\begin{align}
  \nu_X &= - \mu \partial_x \left(U(x,y)+ \beta^{-1} \ln{P_{X,Y}(x,y;t)}\right),\\
  \nu_Y &= - \mu \partial_y \left(U(x,y)+ \beta^{-1} \ln{P_{X,Y}(x,y;t)}\right),
\end{align}
となるので,平衡の条件\eqref{eq}から,平衡分布$P_{X,Y}^{\mathrm{eq}}$に対して
\begin{equation}
  \partial_x \left(U(x,y)+ \beta^{-1} \ln{P_{X,Y}^{\mathrm{eq}}(x,y)}\right) = \partial_y \left(U(x,y)+ \beta^{-1} \ln{P_{X,Y}^{\mathrm{eq}}(x,y)}\right) = 0
\end{equation}
が成り立つ.
よって,$U+\beta^{-1}\ln{P_{X,Y}^{\mathrm{eq}}}$は$(x,y,t)$のいずれにもよらない関数となるので,
\begin{equation}
  U(x,y)+ \beta^{-1} \ln{P_{X,Y}^{\mathrm{eq}}(x,y)} = \mathrm{const.} =: c 
\end{equation}
となる.
したがって,
\begin{equation}
  P_{X,Y}^{\mathrm{eq}}(x,y) = e^c e^{-\beta U(x,y)}
\end{equation}
が成り立つ.
この両辺を$x,y$について積分すると,規格化条件より,
\begin{align}
  1 &= e^c \int \dd{x} \int \dd{y} e^{-\beta U(x,y)} \notag \\
  \therefore e^c &= \left[\int \dd{x} \int \dd{y} e^{-\beta U(x,y)} \right]^{-1}
\end{align}
となる.
具体的な$U$の表式を代入して計算すると,
\begin{equation}
  \int \dd{x} \int \dd{y} e^{-\beta U(x,y)} = \frac{2\pi}{\beta k}
\end{equation}
となるので,
\begin{equation}
  e^c = \frac{\beta k}{2\pi}
\end{equation}
が分かる.
よって,平衡分布は
\begin{equation}
  P_{X,Y}^{\mathrm{eq}}(x,y) = \frac{\beta k}{2\pi} e^{-\frac{\beta k}{2}\left(x^2+y^2\right)}
\end{equation}
と求まる.

\subsection*{1.6}
解けなかったが,分かったところまで書く.

定常分布を$P_{X,Y}^{\mathrm{st}}$とおき,さらに
\begin{align}
  \nu_X^{\mathrm{st}} &:= \mu F_X - \beta^{-1}\partial_x \ln{P_{X,Y}^{\mathrm{st}}} = \mu (-kx+ay) - \mu\beta^{-1}\partial_x \ln{P_{X,Y}^{\mathrm{st}}} \\
  \nu_Y^{\mathrm{st}} &:= \mu F_Y - \beta^{-1}\partial_y \ln{P_{X,Y}^{\mathrm{st}}} = \mu (-ax-ky) - \mu\beta^{-1}\partial_y \ln{P_{X,Y}^{\mathrm{st}}}
\end{align}
とおくと,
\begin{equation}
  \partial_x(\nu_X^{\mathrm{st}} P_{X,Y}^{\mathrm{st}}) + \partial_x(\nu_X^{\mathrm{st}} P_{X,Y}^{\mathrm{st}}) = 0
\end{equation}
が成り立つ.
代入して整理すると,
\begin{equation}
  (\partial_x^2 + \beta kx\partial_x) P_{X,Y}^{\mathrm{st}} +(\partial_y^2 + \beta ky\partial_y) P_{X,Y}^{\mathrm{st}}= -2\beta k P_{X,Y}^{\mathrm{st}} - \beta a(x\partial_y - y\partial_x)P_{X,Y}^{\mathrm{st}}
\end{equation}
となる.
ここで,関数$F(x,y)$を
\begin{equation}
  P_{X,Y}^{\mathrm{st}} = e^{-\frac{\beta k}{2} (x^2+y^2)} F(x,y)
\end{equation}
とおいて,上式を整理すると,
\begin{equation}
  (\partial_x^2 + \partial_y^2) F(x,y) = - \beta a(x\partial_y - y\partial_x) F(x,y)
\end{equation}
が得られる.
さらに,直交座標$(x,y)$の代わりに極座標$(r,\theta)$を用いると,
\begin{equation}
  F(r,\theta) := F(x=r\cos{\theta}, y=r\sin{\theta})
\end{equation}
として,
\begin{equation}
  \left(\partial_r^2 + \frac{1}{r}\partial_r + \frac{1}{r^2}\partial_{\theta}^2\right) F(r,\theta) = - \beta a\partial_{\theta} F(r,\theta)
\end{equation}
が分かる.
これを変数分離して解こうとしたがうまくいかなかった.

問題の解答にはならないが,確率の非負性を満たさない定常分布は次のように求まった.
問題を簡単にするため
\begin{equation}
  \partial_{\theta} F(r,\theta) = 0
\end{equation}
と仮定すると,
\begin{equation}
  \left(\partial_r^2 + \frac{1}{r}\partial_r \right) F(r,\theta) = 0
\end{equation}
となる.
この解は任意定数$c$を用いて
\begin{equation}
  F(r,\theta) = c \ln{r}
\end{equation}
と書ける.
したがって,
\begin{equation}
  P_{X,Y}^{\mathrm{st}} = c e^{-\frac{\beta kr^2}{2}}\ln{r} 
\end{equation}
と書ける.
これは確率の非負性を満たさないが,規格化条件を満たすように$c$を
\begin{equation}
  c= \left[\int_0^{\infty} e^{-\frac{\beta kr^2}{2}}\ln{r} \cdot 4\pi r^2 \dd{r}\right]^{-1}
\end{equation}
ととることはできる.
この積分は,$u=\beta k r^2/2$と変数変換し,積分の公式を用いると,
\begin{align}
  \int_0^{\infty} e^{-\frac{\beta kr^2}{2}}\ln{r} \cdot 4\pi r^2 \dd{r} 
  &= \frac{2\sqrt{2}\pi}{(\beta k)^{3/2}} \left[\ln{\frac{2}{\beta k}}\cdot\int_0^{\infty} u^{1/2}e^{-u} \dd{u} + \int_0^{\infty} u^{1/2} e^{-u} \ln{u} \dd{u} \right] \notag\\
  &= \frac{2\sqrt{2}\pi}{(\beta k)^{3/2}} \left[\frac{\sqrt{\pi}}{2}\ln{\frac{2}{\beta k}} + \frac{\sqrt{\pi}}{2} (2 -\gamma - \ln{4}) \right] \notag\\
  &= \sqrt{\frac{2\pi^3}{(\beta k)^3}} \left[2 -\gamma - \ln{(2\beta k)} \right]
\end{align}
と書き直せる\footnote{実際にはWolfram Alphaを用いた.}.
ただし,$\gamma$はオイラー定数である.
以上より,確率の非負性を満たさない定常分布
\begin{equation}
  P_{X,Y}^{\mathrm{st}} = \frac{1}{2 -\gamma - \ln{(2\beta k)}}\sqrt{\frac{(\beta k)^3}{2\pi^3}}  e^{-\frac{\beta k}{2}(x^2+y^2)}\ln{\sqrt{x^2+y^2}} 
\end{equation}
が得られる.

\end{document}