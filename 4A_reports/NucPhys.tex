\documentclass[a4paper,11pt]{jsarticle}


% 数式
\usepackage{amsmath,amsfonts}
\usepackage{bm}
\usepackage{physics}
% 画像
\usepackage[dvipdfmx]{graphicx}
% ローマ数字
\usepackage{otf}
% 単位
\usepackage{siunitx}
% 表
\usepackage{multirow}
% 化学反応
\usepackage[version=4]{mhchem}


\begin{document}

\title{Term-paper for Nuclear Physics}
\author{05-211525 Shunichi SAITO}
\date{\today}
\maketitle

I choose the option C.

\section*{1}

(b), (d), and (c) are absolutely forbidden.

First, (b) does not conserve baryon number, which is one on the left side and zero on the right side. 
(b) does not conserve lepton number, which is one on the left side and zero on the right side.

Second, (d) does not conserve the electric charge, which is zero on the left side and 
two elementary electron charge on the right side.

Third, (e) does not conserve baryon number, which is one on the left side and negative one on the right side.


\section*{2}

The exchange operator of two nucleons $P$ changes the sign of the wave function of the whole system, because each nucleon is a fermion.
This operator $P$ is the product of three operator: the position exchange operator $P_R$, the spin exchange operator $P_S$, and the isospin exchange operator $P_T$.
$P_R$ causes a sign change in the orbital momentum $L$.
The sign change is following:
\begin{align}
  P_R \ket{L, S, T} &= (-1)^{L} \ket{L, S, T} \\
  P_S \ket{L, S, T} &= (-1)^{S+1} \ket{L, S, T} \\
  P_T \ket{L, S, T} &= (-1)^{T+1} \ket{L, S, T}.
\end{align}
Thus, the total sign change is $(-1)^{L+S+T}$, while the sign change caused by $P$ is $(-1)$.
Therefore, $L+S+T$ should be odd.

\section*{3}

\begin{equation}
  \mathbf{J} = \mathbf{L} + \mathbf{S}
\end{equation}

\begin{equation}
  \mathbf{L}\cdot \mathbf{S} = \frac{\mathbf{J}^2 - \mathbf{L}^2 - \mathbf{S}^2}{2} = \frac{J(J+1)-L(L+1)-S(S+1)}{2},
\end{equation}
where $J,L,S$ are the eigenvalues of the operators $\mathbf{J,L,S}$, respectively. 

If $S=1, J=L+1$, then
\begin{equation}
  \mathbf{L}\cdot\mathbf{S} = \frac{(L+1)(L+2-L)-2}{2} = L,
\end{equation}
thus,
\begin{equation}
  V(r) = V_1(r) + L V_2(r) + L^2 V_3(r).
\end{equation}

If $S=1, J=L-1$, then
\begin{equation}
  \mathbf{L}\cdot\mathbf{S} = \frac{(L-1-(L+1))-2}{2} = -L-1,
\end{equation}
thus,
\begin{equation}
  V(r) = V_1(r) + (L+1) V_2(r) + (L+1)^2 V_3(r).
\end{equation}

\section*{4}
\begin{equation}
  S_{12} = 3\left(\bm{\sigma}_1\cdot \frac{\bm{r}}{r}\right)\left(\bm{\sigma}_2\cdot \frac{\bm{r}}{r}\right) - \left(\bm{\sigma}_1 \bm{\sigma}_2\right)
\end{equation}

For $k=1,2$,
\begin{equation}
  \left(\bm{\sigma}_k \cdot \frac{\bm{r}}{r}\right)^2 = \sigma_k^i\frac{r^i}{r}\sigma_k^j\frac{r^j}{r} = \frac{\sigma_k^i\sigma_k^j+\sigma_k^j\sigma_k^i}{2}\frac{r^ir^j}{r^2} = \frac{2\delta^{ij}}{2}\frac{r^ir^j}{r^2} = 1.
\end{equation}
Thus, by using
\begin{equation}
  \bm{S} = \bm{s}_1 + \bm{s}_2 = \frac{\bm{\sigma}_1 + \bm{\sigma}_2}{2},
\end{equation}
\begin{align}
  \left(\bm{S}\cdot \frac{\bm{r}}{r}\right)^2 &= \frac{1}{4}\left(\bm{\sigma}_1\cdot \frac{\bm{r}}{r}\right)^2 + \frac{1}{4}\left(\bm{\sigma}_2\cdot \frac{\bm{r}}{r}\right)^2 + \frac{1}{2}\left(\bm{\sigma}_1\cdot \frac{\bm{r}}{r}\right)\left(\bm{\sigma}_2\cdot \frac{\bm{r}}{r}\right) \notag\\
  &= \frac{1}{2}\left[1+\left(\bm{\sigma}_1\cdot \frac{\bm{r}}{r}\right)\left(\bm{\sigma}_2\cdot \frac{\bm{r}}{r}\right)\right]
\end{align}
and
\begin{equation}
  \bm{S}^2 = \frac{1}{4}\bm{\sigma}_1^2 + \frac{1}{4}\bm{\sigma}_2^2 + \frac{1}{2}\bm{\sigma}_1\cdot \bm{\sigma}_2 = \frac{3}{2} + \frac{1}{2}\bm{\sigma}_1\cdot \bm{\sigma}_2
\end{equation}
are derived.
Therefore, 
\begin{equation}
  S_{12} = 2\left[3\left(\bm{S}\cdot \frac{\bm{r}}{r}\right)^2 - \bm{S}^2\right].
\end{equation}

\section*{5}

Suppose Coulomb force is dominant.
The smallest distance $r$ to the Pb center that proton can reach satisfies
\begin{equation}
  \frac{\alpha \hbar c Z }{r} = \SI{10}{\MeV},
\end{equation}
where $\alpha=1/137$ is the fine-structure constant, $Z=82$ is the atomic number of Pb.
Thus, 
\begin{equation}
  r = \frac{\alpha\hbar c Z}{\SI{10}{\MeV}} \approx \SI{12}{fm}.
\end{equation}
On the other hand, the radius of Pb atom $R$ satisfies
\begin{equation}
  R = r_0 A^{1/3} \approx \SI{7}{fm},
\end{equation}
where $r_0=\SI{1.2}{fm}$ is determined experimentally and $A=208$ is the mass number of Pb.
Therefore, the smallest distance to the Pb surface is 
\begin{equation}
  r - R = \SI{5}{fm}.
\end{equation}

\section*{6}
According to Live Chart of Nuclides by IAEA (International Atomic Energy Agency), the energies required for \ce{^238 U} decay are
\begin{align}
  Q_{\beta^-} &= -\SI{146.912}{\keV}, \\
  Q_{\alpha} &= \SI{4269.921}{\keV}, \\
  Q_{\mathrm{EC}} &= -\SI{358616}{\keV}, \\
  S_n &= \SI{6153.713}{\keV}, \\
  S_p &= \SI{750913}{\keV},
\end{align}
where $Q_{\beta^-}, Q_{\alpha}$ and $Q_{\mathrm{EC}}$ are the energies available for $\beta^-$ decay, $\alpha$ decay, and EC decay, respectively, and $S_n$ and $S_p$ are the separation energies to remove one neutron and one proton from \ce{^238 U}, respectively.
Thus, the energies required for decay by $p, n, e^-$ and $e^+$ emission are $S_p, S_n, -Q_{\beta^-}$ and $-Q_{\mathrm{EC}}$, respectively, and all of them are positive values greater than \SI{1}{\MeV}.
Therefore, \ce{^238 U} is stable with respect to these decay.
On the other hand, the energy required for decay by $\alpha-$particle is $-Q_{\alpha}$, which is a negative value.
Therefore, \ce{^238 U} is unstable with respect to $\alpha$ decay.

\section*{7}
As discussed in the next problem, in the Fermi gas model, the average kinetic energy of nuclear is calculated to be
\begin{equation}
  E_T \approx 2\times \SI{32}{\MeV}\times A^{-2/3}\left(\frac{A}{2}\right)^{2/3} \approx \SI{20}{\MeV} \times A,
\end{equation}
when $N=Z=A/2$.
Therefore, one nucleon has an average kinetic energy of \SI{20}{\MeV}.

\section*{8}
In the Fermi gas model, the density of nucleons in the nuclear is given by 
\begin{equation}
  \rho = \sum_{\vec{p}}^{\mathrm{occ.}} (\mathrm{number\,of\,states}) = \frac{2}{(2\pi)^3} \int_0^{p_F} \mathrm{d}^3 p = \frac{1}{3\pi^2} p_F^3.
\end{equation}
Thus, if $N=Z=A/2$, the number of nucleons is 
\begin{equation}
  \frac{A}{2} = \rho V = \frac{p_F^3}{3\pi^2} \frac{4\pi r_0^3 A}{3},
\end{equation}
where $r_0\approx \SI{1.2}{fm}$ is the radius of a liquid drop.
It reduces to
\begin{equation}
  p_F = \frac{1}{r_0}\left(\frac{9\pi A/2}{4A}\right)^{1/3}.
\end{equation}
Therefore, the kinetic energy of each nucleons is
\begin{equation}
  \ev{E^{\mathrm{kin}}} = \frac{\int_0^{p_F} \frac{p^2}{2M}\mathrm{d}^3 p}{\int_0^{p_F}\mathrm{d}^3 p} = \frac{3}{5}\frac{p_F^2}{2M}
\end{equation}
and the total kinetic energy is 
\begin{align}
  E_T 
  &= \frac{A}{2} \ev{E^{\mathrm{kin}}} \times 2 = \frac{3A/2}{10M}\frac{1}{r_0^2}\left(\frac{9\pi A/2}{4A}\right)^{2/3} \times 2 \notag\\
  &= 2\frac{3}{10Mr_0^2}\left(\frac{9\pi}{4}\right)^{2/3} A^{-2/3}\left(\frac{A}{2}\right)^{5/3} \notag \\
  &= 2C_3A^{-2/3}\left(\frac{A}{2}\right)^{5/3},
\end{align}
where
\begin{equation}
  C_3 = \frac{3}{10Mr_0^2}\left(\frac{9\pi}{4}\right)^{2/3} \approx \SI{32}{\MeV}.
\end{equation}

If $N\neq Z$, then the kinetic energy of protons and neutrons are 
\begin{equation}
  \ev{E^{\mathrm{kin}}}_p = \frac{3}{5}\frac{p_{F,p}^2}{2M}, \qquad \ev{E^{\mathrm{kin}}}_n = \frac{3}{5}\frac{p_{F,n}^2}{2M},
\end{equation}
respectively, where
\begin{equation}
  p_{F,p} = \frac{1}{r_0}\left(\frac{9\pi Z}{4A}\right)^{1/3}, \qquad p_{F,n} = \frac{1}{r_0}\left(\frac{9\pi N}{4A}\right)^{1/3},
\end{equation}
and $M$ is the mass of proton and neutron (assumed they have the same mass).
Therefore, the total kinetic energy is 
\begin{align}
  E_T' &= Z\ev{E^{\mathrm{kin}}}_p + N\ev{E^{\mathrm{kin}}}_n 
  = \frac{3}{10M}\left(Zp_{F,p}^2 + Np_{F,n}^2\right) \notag\\
  &= \frac{3}{10M}\frac{1}{r_0^2} \left(\frac{9\pi}{4}\right)^{2/3} \frac{N^{5/3}+Z^{5/3}}{A^{2/3}} \notag\\
  &= C_3 A^{-2/3}\left(N^{5/3}+Z^{5/3}\right).
\end{align}

\section*{9}

In the case of
\begin{equation}
  \ce{^40_20 Ca -> ^36_18 Ar + ^4_2He},
\end{equation}
the height of the Coulomb barrier is
\begin{equation}
  V_c = \frac{\alpha\hbar c}{r_0}\frac{Z_1Z_2}{A_1^{1/3}+A_2^{1/3}} = \frac{1/137\times \SI{200}{\MeV fm}}{\SI{1.2}{fm}}\frac{18\times 2}{36^{1/3}+4^{1/3}} \approx \SI{9.0}{\MeV}.
\end{equation}
In the case of
\begin{equation}
  \ce{^112_50 Sn -> ^108_48 Cd + ^4_2He},
\end{equation}
the height of the Coulomb barrier is
\begin{equation}
  V_c = \frac{\alpha\hbar c}{r_0}\frac{Z_1Z_2}{A_1^{1/3}+A_2^{1/3}} = \frac{1/137\times \SI{200}{\MeV fm}}{\SI{1.2}{fm}}\frac{48\times 2}{108^{1/3}+4^{1/3}} \approx \SI{18}{\MeV}.
\end{equation}
In the case of
\begin{equation}
  \ce{^232_90 Th -> ^228_88 Cd + ^4_2He},
\end{equation}
the height of the Coulomb barrier is
\begin{equation}
  V_c = \frac{\alpha\hbar c}{r_0}\frac{Z_1Z_2}{A_1^{1/3}+A_2^{1/3}} = \frac{1/137\times \SI{200}{\MeV fm}}{\SI{1.2}{fm}}\frac{48\times 2}{232^{1/3}+4^{1/3}} \approx \SI{28}{\MeV}.
\end{equation}

\section*{10}

\subsection*{(a)}

E1 transition occurs in the following reactions:
\begin{equation}
  (5/2)^{-}\to (3/2)^{+},\qquad (5/2)^{-}\to (7/2)^{+},\qquad (1/2)^{-}\to (3/2)^{+},\qquad (3/2)^{-}\to (3/2)^{+}.
\end{equation}
Hence, $(3/2)^{+}$ can be an isomeric state.

\subsection*{(b)}

M1 transition occurs in 
\begin{equation}
  (5/2)^{-} \to (3/2)^{-},\qquad (3/2)^{-}\to (1/2)^{-},
\end{equation}
E2 transition occurs in
\begin{equation}
  (5/2)^{-} \to (1/2)^{-}, \qquad (3/2)^{+}\to (7/2)^{+}
\end{equation}
and M2 transition occurs in
\begin{equation}
  (3/2)^{-}\to (7/2)^{+}.
\end{equation}
Therefore, the possibility of reaction
\begin{equation}
  (1/2)^{-}\to (7/2)^{+}
\end{equation} 
is the smallest.
This reaction is E3 transition.

\end{document}