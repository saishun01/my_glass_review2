\documentclass[a4paper,11pt]{jsarticle}


% 数式
\usepackage{amsmath,amsfonts}
\usepackage{bm}
\usepackage{physics}
% 画像
\usepackage[dvipdfmx]{graphicx}
% ローマ数字
\usepackage{otf}
% 単位
\usepackage{siunitx}
% 表
\usepackage{multirow}
% 化学反応
\usepackage[version=4]{mhchem}


\begin{document}

\title{解析学XCレポート}
\author{05-211525 齋藤駿一}
\date{\today}
\maketitle

\section*{1}
\begin{equation}
  L = \left(
    \begin{matrix}
      l_1 & l_2 \\
      l_2 & l_3 
    \end{matrix}
  \right)
\end{equation}
に対し,
\begin{equation}
  \dv{x(t)}{t} = Lx(t),\qquad t>0,\qquad x(0)>a
\end{equation}
を満たすとする.
ただし,$l_1,l_2,l_3$を実数として,$l_2\neq 0$とする.

\subsection*{(1)}
$L$の固有値$\lambda$に対し,固有方程式
\begin{equation}
  \left|L-\lambda I\right| = (l_1-\lambda)(l_3-\lambda)-l_2^2 = 0
\end{equation}
すなわち
\begin{equation}
  \lambda^2 - (l_1+l_3)\lambda + l_1l_3-l_2^2 = 0 \label{eigeq}
\end{equation}
を満たす.
これを$\lambda$に関する2次方程式と見たとき,判別式$D$は
\begin{equation}
  D = (l_1+l_3)^2 - 4(l_1l_3-l_2^2) = (l_1-l_3)^2 + 4l_2^2 > 0
\end{equation}
となる.
ここで$l_2\neq 0$を用いた.
したがって,固有方程式\eqref{eigeq}は相異なる二つの実数解を持つ.
つまり,$L$は実数の相異なる固有値$\lambda_1,\lambda_2$を持つ.

\subsection*{(2)}
$i=1,2$に対して,固有値$\lambda_i$に対する固有ベクトル$v_i= ^t(v_{i1},v_{i2})$について
\begin{equation}
  (L-\lambda_i I)v_i = \left(
    \begin{matrix}
      l_1 - \lambda_i & l_2 \\
      l_2 & l_3 - \lambda_i
    \end{matrix}
  \right)
  \left(
    \begin{matrix}
      v_{i1} \\
      v_{i2}
    \end{matrix}
  \right)
\end{equation}
が成り立つ.
これより
\begin{equation}
  v_{i2} = - \frac{l_1-\lambda_i}{l_2} v_{i1}
\end{equation}
であり,$v_i$が規格化されているとすると,
\begin{equation}
  1 = v_{i1}^2 + v_{i2}^2 = \frac{(l_1-\lambda_i)^2+l_2^2}{l_2^2} v_{i1}^2
\end{equation}
が成り立つことから,
\begin{equation}
  v_i = \frac{1}{\sqrt{(l_1-\lambda_i)^2+l_2^2}}
  \left(
  \begin{matrix}
    l_2 \\
    -(l_1-\lambda_i)
  \end{matrix}
  \right)
\end{equation}
と書ける.
これを用いると,$P_i\,(i=1,2)$は
\begin{equation}
  P_i = v_i ^t v_i = \frac{1}{(l_1-\lambda_i)^2+l_2^2}
  \left(
  \begin{matrix}
    l_2^2 & -l_2(l_1-\lambda_i) \\
    -l_2(l_1-\lambda_i) & (l_1-\lambda_i)^2
  \end{matrix}
  \right)
\end{equation}
と分かる.
固有値$\lambda_i$は,固有方程式\eqref{eigeq}を解いて,
\begin{equation}
  \lambda_1 = \frac{(l_1+l_3)+\sqrt{(l_1-l_3)^2+4l_2^2}}{2},\qquad \lambda_2 = \frac{(l_1+l_3)-\sqrt{(l_1-l_3)^2+4l_2^2}}{2}
\end{equation}
と分かるので,これを代入することで$P_i$を$l_1,l_2,l_3$を用いて表せる.

\subsection*{(3)}
射影演算子の性質を用いると,
\begin{align}
  e^{Lt} &= \sum_{n=0}^{\infty} \frac{L^nt^n}{n!} 
  = \sum_{n=0}^{\infty} \frac{t^n}{n!}(\lambda_1 P_1 + \lambda_2 P_2)^n \notag\\
  &= \sum_{n=0}^{\infty} \frac{t^n}{n!} (\lambda_1^n P_1 + \lambda_2^n P_2) 
  = \left(\sum_{n=0}^{\infty} \frac{(\lambda_1t)^n}{n!}\right)P_1 + \left(\sum_{n=0}^{\infty} \frac{(\lambda_2t)^n}{n!}\right)P_2 \notag\\
  &= e^{\lambda_1t}P_1 + e^{\lambda_2t}P_2 
\end{align}
と計算できる.

\subsection*{(4)}
$x(t) = e^{Lt}a$とおくと,$x(0)=a$が成り立ち,
\begin{align}
  \dv{x(t)}{t} &= \left(\dv{}{t} e^{Lt}\right)a  \notag\\
  &= \dv{}{t} \left(e^{\lambda_1t}P_1+e^{\lambda_2t}P_2\right)a \notag\\
  &= (\lambda_1e^{\lambda_1t}P_1 + \lambda_2e^{\lambda_2t}P_2)a \notag\\
  &= (\lambda_1P_1+\lambda_2P_2)(e^{\lambda_1t}P_1+e^{\lambda_2t}P_2) a \notag\\
  &= Le^{Lt}a = Lx(t)
\end{align}
も成り立つ.

\subsection*{(5)}
\begin{equation}
  x(t) = (e^{\lambda_1 t}P_1 + e^{\lambda_2 t}P_2)a
\end{equation}
において,$t=T$の場合($x(T)=b$)を考えると,
\begin{equation}
  b = (e^{\lambda_1 T}P_1 + e^{\lambda_2 T}P_2)a
\end{equation}
となる.
ここに前問(2)で求めた$P_i$を代入し整理しようとしたが,式が煩雑になって$l_1,l_2,l_3$について解けなかった.

\subsection*{(6)}
$l_1=l_3=0$のとき,$L$の固有値は$\lambda_1=l_2,\,\lambda_2=-l_2$となり,スペクトル分解は
\begin{equation}
  L = 
  \frac{l_2}{2}\left(
  \begin{matrix}
    1 & 1 \\
    1 & 1 
  \end{matrix}
  \right) + 
  \frac{-l_2}{2}\left(
  \begin{matrix}
    1 & -1 \\
    -1 & 1
  \end{matrix}
  \right)
\end{equation}
となる.
よって,解は
\begin{align}
  x(t) &= (e^{l_2 t}P_1 + e^{-l_2 t}P_2)a 
  = \left(\frac{e^{l_2 t}}{2}\left(
  \begin{matrix}
    1 & 1 \\
    1 & 1 
  \end{matrix}
  \right) + 
  \frac{e^{-l_2 t}}{2}\left(
  \begin{matrix}
    1 & -1 \\
    -1 & 1
  \end{matrix}
  \right)\right) a \notag\\
  &= \left(
    \begin{matrix}
      \cosh{(l_2 t)} & \sinh{(l_2 t)} \\
      \sinh{(l_2 t)} & \cosh{(l_2 t)}
    \end{matrix}
    \right)a
\end{align}
と書ける.
ここに$x(T)=b$を代入して成分表示すると,
\begin{equation}
  \left(
    \begin{matrix}
      b_1 \\
      b_2
    \end{matrix}
  \right)
  =
  \left(
    \begin{matrix}
      \cosh{(l_2 T)} & \sinh{(l_2 T)} \\
      \sinh{(l_2 T)} & \cosh{(l_2 T)}
    \end{matrix}
  \right)
  \left(
    \begin{matrix}
      a_1 \\
      a_2
    \end{matrix}
  \right)
  = 
  \left(
    \begin{matrix}
      a_1 \cosh{(l_2 T)} + a_2 \sinh{(l_2 T)}\\
      a_1 \sinh{(l_2 T)} + a_2 \cosh{(l_2 T)} 
    \end{matrix}
  \right)
\end{equation}
となり,これより
\begin{align}
  b_1 + b_2 &= (a_1 + a_2)e^{l_2 T} \\
  b_1 - b_2 &= (a_1 - a_2)e^{-l_2 T}
\end{align}
が成り立つ.
よって,$b_1,b_2$について
\begin{equation}
  b_1^2 - b_2^2 = (b_1+b_2)(b_1-b_2) = (a_1+a_2)(a_1-a_2) = a_1^2 - a_2^2 \label{ab}
\end{equation}
が成り立ち,$l_2$を推定する式として,
\begin{equation}
  l_2 = 
  \begin{cases}
    \frac{1}{T} \log{\left|\frac{b_1+b_2}{a_1+a_2}\right|} & (a_1\neq -a_2,\quad b_1\neq -b_2)\\
    \frac{1}{T} \log{\left|\frac{a_1-a_2}{b_1-b_2}\right|} & (a_1\neq a_2,\quad b_1\neq b_2)
  \end{cases}
\end{equation}
が得られる.
これより,条件\eqref{ab}を満たすとき,$l_2$は一意に定まる.

\section*{2}
\subsection*{(1)}
運動方程式
\begin{equation}
  \dv[2]{x(t)}{t} = -g,\qquad x(0)=h,\qquad \dv{x(0)}{t}=0
\end{equation}
を解くと,
\begin{equation}
  x(t) = h - \frac{1}{2}gt^2
\end{equation}
となる.
時刻$t=t_0$の情報を代入すると
\begin{equation}
  x(t_0) = h - \frac{1}{2}gt_0^2
\end{equation}
となり,これを整理すると
\begin{equation}
  g = \frac{2(h-x(t_0))}{t_0^2}
\end{equation}
が得られる.

\subsection*{(2)}
運動方程式
\begin{equation}
  \dv[2]{x(t)}{t} = -k\dv{x(t)}{t} - g,\qquad x(0)=h,\qquad \dv{x(0)}{t}=0
\end{equation}
を解く.
まず,
\begin{equation}
  \dv{}{t} \left(e^{kt}\dv{x(t)}{t}\right) = \left(\dv[2]{x(t)}{t}+k\dv{x(t)}{t}\right)e^{kt} = -ge^{kt}
\end{equation}
の両辺を時刻$t=0$から$t$まで積分すると,初期条件から
\begin{equation}
  e^{kt}\dv{x(t)}{t} = \int_0^t (-ge^{kt})\dd{t} = -\frac{g}{k}(e^{kt}-1)
\end{equation}
すなわち
\begin{equation}
  \dv{x(t)}{t} = -\frac{g}{k}(1-e^{-kt})
\end{equation}
が得られる.
この両辺を再び時刻$t=0$から$t$まで積分すると,初期条件から
\begin{equation}
  x(t) - h = - \frac{g}{k}\int_0^t (1-e^{-kt})\dd{t} = - \frac{g}{k} \left[t+\frac{e^{-kt}}{k}\right]_{0}^{t} = - \frac{g}{k} \left(t+\frac{e^{-kt}-1}{k}\right)
\end{equation}
が得られる.
ここに時刻$t=t_0$の情報を代入すると,
\begin{align}
  x(t_0) - h &= - \frac{g}{k} \left(t_0+\frac{e^{-kt_0}-1}{k}\right) \notag \\
  \frac{h-x(t_0)}{gt_0^2} &= \frac{1}{kt_0}\left(1+\frac{e^{-kt_0}-1}{kt_0}\right)
\end{align}
となる.
そこで関数$f$を
\begin{equation}
  f(y) = \frac{1}{y}\left(1+\frac{e^{-y}-1}{y}\right) = \frac{e^{-y}+y-1}{y^2} \label{f}
\end{equation}
と定義すると,
\begin{equation}
  \frac{h-x(t_0)}{gt_0^2} = f(kt_0)
\end{equation}
と書ける.
ここで
\begin{equation}
  f'(y) = -\frac{(y+2)}{y^3}\left(e^{-y}+\frac{y-2}{y+2}\right)
\end{equation}
であり,これは$y>0$で常に負であることが次のようにして示せる.

まず,$y\ge 2$のとき$f'(y)<0$は分かる.
次に,$0<y<2$に対して
\begin{equation}
  g(y) = y - \ln{(y+2)} + \ln{(2-y)}
\end{equation}
を定義すると,
\begin{equation}
  g'(y) = 1 - \frac{1}{2+y} + \frac{-1}{2-y} = -\frac{y^2}{4-y^2} < 0
\end{equation}
が成り立つので,$g(y)$は単調に減少する.
これと$g(0)=0$より,$g(y)<0$がいえる.
したがって,$0<y<2$に対して
\begin{align}
  &y < \ln{\left(\frac{2+y}{2-y}\right)} \\
  &-y < \ln{\left(\frac{2-y}{2+y}\right)} \\
  &e^{-y} - \frac{2-y}{2+y} > 0
\end{align}
がいえる.
よって,$0<y<2$に対して$f'(y)<0$が分かる.
以上より,$y>0$で常に$f'(y)<0$である.

したがって,$y>0$で$f(y)$は単調に減少するので,$y>0$で$f$の逆関数$f^{-1}$を定義できる.
これを用いると,
\begin{equation}
  kt_0 = f^{-1}\left(\frac{h-x(t_0)}{gt_0^2}\right)
\end{equation}
すなわち
\begin{equation}
  k = \frac{1}{t_0}f^{-1}\left(\frac{h-x(t_0)}{gt_0^2}\right)
\end{equation}
が成り立つ.

\section*{3}
\subsection*{(1)}
\begin{equation}
  E(t) = \int_0^1 |u(x,t)|^2 dx
\end{equation}
に対して,
\begin{align}
  E'(t) &= 2\int_0^1 u(x,t) u_t(x,t) dx \\
  &= 2\int_0^1 u(x,t) u_{xx}(x,t) dx = \left[2u(x,t)u_x(x,t)\right]_0^1 - 2 \int_0^1 |u_x(x,t)|^2 dx \notag\\
  &= - 2 \int_0^1 |u_x(x,t)|^2 dx
\end{align}
および
\begin{align}
  E''(t) &= \dv{}{t} \left(- 2 \int_0^1 |u_x(x,t)|^2 dx\right) \notag\\
  &= - 4 \int_0^1 u_x(x,t)u_{xt}(x,t) dx = - 4 \int_0^1 u_x(x,t)u_{xxx}(x,t) dx \notag\\
  &= -\left[4u_x(x,t)u_{xx}(x,t)\right]_0^1 + 4\int_0^1 |u_{xx}(x,t)|^2 dx = 4\int_0^1 |u_{xx}(x,t)|^2 dx
\end{align}
が分かる.
ここで,
\begin{equation}
  u_{xx}(0,t) = u_{t}(0,t) = 0,\qquad u_{xx}(1,t) = u_{t}(1,t) = 0,
\end{equation}
を用いた.
したがって,Cauchy-Schwarz不等式
\begin{equation}
  \left(\int f(x) g(x) dx\right)^2 \le \int f(x)dx \int g(x)dx
\end{equation}
を$f(x)=w(x,t),\,g(x)=2w_x(x,t)$の場合に適用すると,
\begin{equation}
  E'(t)^2 \le E(t) E''(t)
\end{equation}
が得られる.
したがって,
\begin{equation}
  E(t)E''(t) - E'(t)^2 \ge 0
\end{equation}
が成り立つ.

\subsection*{(2)}
$t\ge 0$に対して常に$E(t)\neq 0$のとき,
\begin{equation}
  (\log{E(t)})'' = \left(\frac{E'(t)}{E(t)}\right)' = \frac{E(t)E''(t) - E'(t)^2}{E(t)^2} \ge 0
\end{equation}
が成り立つので,$\log{E(t)}$は上に凸である.
したがって,$t=0,t_0,T\,(0<t_0<T)$における$E(t)$の値に関して,凸不等式と$E(0)\le M$から
\begin{equation}
  \log{E(t_0)} \le \frac{(T-t_0)\log{E(0)} + t_0\log{E(T)}}{T} \le \frac{(T-t_0)\log{M} + t_0\log{E(T)}}{T}
\end{equation}
が成り立つ.
よって,$E(t_0)$は
\begin{equation}
  E(t_0) \le M^{(T-t_0)/T} E(T)^{t_0/T} = M \left(\frac{E(T)}{M}\right)^{t_0/T}
\end{equation}
と上から抑えられる.

\subsection*{(3)}
ある時刻$0\le t' \le T$で$E(t')=0$となる場合,$\log{E(t')}$が定義できないため,上の議論が成り立たない.
この場合,$E'(t)\le 0$より$0\le E(T)\le E(t')=0$より$E(T)=0$がいえる.
そのため,時刻$0<t_0<T$に対し,$E(t_0)$を$E(T)$を用いて評価することはできないが,$E(0)\le M$を用いた評価$E(t_0)\le E(0) \le M$は成り立つ.

\section*{4}
2つの問題を考える.
\begin{align}
  u_t(x,t) = u_{xx}(x,t) + \mu(t)f(x),\qquad 0<x<1,\,t>0 \\
  u(0,t)=u(1,t)=0,\quad t>0,\qquad u(x,0)=0,\quad 0<x<1.
\end{align}
と
\begin{align}
  v_t(x,t) = v_{xx}(x,t),\qquad 0<x<1,\,t>0 \\
  v(0,t)=v(1,t)=0,\quad t>0,\qquad v(x,0)=f(x),\quad 0<x<1.  
\end{align}

\subsection*{(1)}
$v(x,t)$をフーリエ変換して
\begin{equation}
  v(x,t) = \int e^{ikx} \tilde{v}(k,t) \dd{k} \label{v_ft1}
\end{equation}
と書くと,
\begin{equation}
  0 = v_t(x,t) - v_{xx}(x,t) = \int e^{ikx} (\tilde{v}_t(k,t) + k^2 \tilde{v}(k,t)) \dd{k} 
\end{equation}
が得られる.
この両辺をフーリエ逆変換することで,
\begin{equation}
  \tilde{v}_t(k,t) = -k^2\tilde{v}_{xx}(k,t)
\end{equation}
が得られる.
これを解くと,
\begin{equation}
  \tilde{v}(k,t) = \tilde{v}(k,0)e^{-k^2t/2}
\end{equation}
となる.
したがって,式\eqref{v_ft1}より
\begin{equation}
  v(x,t) = \int e^{ikx} \tilde{v}(k,0)e^{-k^2t/2} \dd{k} \label{v_ft2}
\end{equation}
がいえる.
$t=0$の場合を考え,
\begin{equation}
  v(x,0) = \int e^{ikx} \tilde{v}(k,0) \dd{k}
\end{equation}
を逆フーリエ変換すると,
\begin{equation}
  \tilde{v}(k,0) = \frac{1}{2\pi}\int e^{-ikx} v(x,0) \dd{x} = \frac{1}{2\pi}\int e^{-ikx} f(x) \dd{x}
\end{equation}
が得られるので,これを式\eqref{v_ft2}に代入すると,
\begin{align}
  v(x,t) &= \frac{1}{2\pi} \int e^{ikx} \int e^{-iky} v(y,0) e^{-k^2t/2}  \dd{y}  \dd{k} \notag\\
  &= \frac{1}{2\pi} \int f(y) \int  e^{ik(x-y)} e^{-k^2t/2} \dd{k} \dd{y} \notag\\
  &= \frac{1}{2\pi} \int f(y) \int \exp{-t\left(k-\frac{i(x-y)}{2t}\right)^2-\frac{(x-y)^2}{4t}}   \dd{k}\dd{y} \notag\\
  &= \frac{1}{2\pi} \int f(y) e^{-\frac{(x-y)^2}{4t}} \sqrt{\frac{\pi}{t}} \dd{y} \notag\\
  &= \sqrt{\frac{1}{4\pi t}} \int  f(y) e^{-\frac{(x-y)^2}{4t}} \dd{y}
\end{align}
が分かる.

\subsection*{(2)}

\begin{equation}
  u(x,t) = \int_0^t  \mu(s)v(x,t-s) ds
\end{equation}
と書けることを示す.
そのために,右辺を$\tilde{u}(x,t)$とし,$u(x,t)=\tilde{u}(x,t)$を示す.

まず,$v(x,0)=0,\,v(0,t)=v(1,t)=0$を用いると,
\begin{equation}
  \tilde{u}(x,0) = 0,\qquad \tilde{u}(0,t) = \tilde{u}(1,t) = 0
\end{equation}
が分かる.
次に,$v(x,0)=f(x)$を用いると
\begin{equation}
  \partial_t \tilde{u}(x,t) =  \mu(t)v(x,0) + \int_0^t \mu(s) \partial_t v(x,t-s) ds = \mu(t)f(x) + \int_0^t \mu(s) \partial_t v(x,t-s) ds
\end{equation}
が分かり,
\begin{equation}
  \partial_{xx} \tilde{u}(x,t) =\int_0^t \mu(s)\partial_{xx} v(x,t-s) ds
\end{equation}
が成り立つので,$\partial_t v(x,t-s)=\partial_{xx} v(x,t-s)$より,
\begin{equation}
  \partial_t \tilde{u}(x,t) = \partial_{xx} \tilde{u}(x,t) + \mu(t)f(x)
\end{equation}
がいえる.
以上より,$\tilde{u}(x,t)$は$u(x,t)$に関する問題の解の一つである.
加えて,$w(x,t)=u(x,t)-\tilde{u}(x,t)$とおくと,これは
\begin{align}
  w_t(x,t) = w_{xx}(x,t),\qquad 0<x<1,\,t>0 \\
  w(0,t)=w(1,t)=0,\quad t>0,\qquad w(x,0)=0,\quad 0<x<1.
\end{align}
の解である.
ここで,
\begin{equation}
  E(t) = \int_0^1 |w(x,t)|^2 dx
\end{equation}
を定義すると,$E(t)\ge 0$である.
一方
\begin{equation}
  \dv{E(t)}{t} =  -2\int_0^1 |w_x(x,t)|^2 dx \le 0 
\end{equation}
と分かるので,$E(t)$は単調非増加である.
いま
\begin{equation}
  E(t) \le E(0) = 0
\end{equation}
より$E(t)=0$,すなわち任意の$0<x<1,t>0$に対して$w(x,t)=0$がいえる.
したがって,$u(x,t)=\tilde{u}(x,t)$である.

\subsection*{(3)}


\end{document}