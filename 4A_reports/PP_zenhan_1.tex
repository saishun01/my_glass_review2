\section*{1}
重心系で見た衝突前の全系のエネルギーを$E_{\mathrm{CM}}$とし,実験室系で見た衝突する陽子のそれぞれのエネルギー,運動量を$E_i,\vec{p}_i\,(i=1,2)$とおくと,
\begin{equation}
  E_{\mathrm{CM}}^2 = (E_1 + E_2)^2 - (m_1 + m_2)^2
\end{equation}
となる.$E_{\mathrm{CM}} \ge 4m_p$が必要な条件である.

\subsection*{(1)}
$\vec{p}_2 = 0$とおくと,$E_2=m_p$と分かる.
よって,$E_1=m_p+K$から,
\begin{align}
  E_{\mathrm{CM}}^2 &= (m_p+K+m_p)^2 - p_1^2 \notag\\
  &= (2m_p+K)^2 - \left[(m_p+K)^2 - m_p^2\right] \notag\\
  &= 4m_p^2 + 2m_pK
\end{align}
よって,反応が起こる条件は
\begin{equation}
  4m_p^2 + 2m_pK \ge (4m_p)^2
\end{equation}
すなわち
\begin{equation}
  K \ge 6m_p \approx \SI{5.63}{\GeV}
\end{equation}
である.

\subsection*{(2)}
$\vec{p}_1+\vec{p}_2=0$より,反応が起こる条件は,
\begin{align}
  E_{\mathrm{CM}}^2 = \left(2(m_p + K)\right)^2 \ge (4m_p)^2
\end{align}
すなわち
\begin{equation}
  K \ge m_p = \SI{938}{\MeV}
\end{equation}
である.

\section*{2}
\subsection*{2.1}
4元運動量$p_1,p_2,p_3$の各々が4つの成分を持つので,系は計12個の変数で記述できる.
ここで,3つの関係式
\begin{equation}
  p_i^2 = m_i^2\qquad i = 1,2,3
\end{equation}
のため,変数は9個に減らせる.
また,崩壊前の粒子の4元運動量を$P$として,エネルギー・運動量保存則
\begin{equation}
  P = p_1 + p_2 + p_3
\end{equation}
が成り立つ.
これは4成分それぞれで成り立つので,変数は5個に減らせる.
さらに,系を適切に回転させることで,常に粒子1がx方向に運動し,粒子2がx-y平面に含まれるようにすることができる.
その結果,粒子1の運動量のy,z成分と粒子2の運動量のz成分の3変数を0にできるので,変数は2個に減らせる.
したがって,系の自由度は2である.

\subsection*{2.2}
以下では,$i=1,2,3$に対して
\begin{equation}
  p_i = \left(
    \begin{array}{c}
      E_i \\
      \vec{p}_i
    \end{array}
  \right)  
\end{equation}
とおく.
まず$m_{12}^2$の動く範囲を考える.
その上でこの3体崩壊を,粒子1と2の複合粒子と粒子3への2体崩壊のように扱う.
まず,複合粒子の運動量を
\begin{equation}
  p_{12} = p_1 + p_2
\end{equation}
と定義すると,エネルギー・運動量保存則は
\begin{equation}
  P = p_{12} + p_3
\end{equation}
と書ける.
ここで,崩壊前の粒子は静止していることから,
\begin{equation}
  P = \left(
    \begin{array}{c}
      M \\
      \vec{0}
    \end{array}
  \right)
\end{equation}
となる.
よって,$P\cdot p_3 = ME_3$が成り立つ.
以上より,
\begin{align}
  m_{12}^2 &= \left(p_1 + p_2\right)^2 \notag\\
  &= \left(P - p_3\right)^2 \notag\\
  &= P^2 + p_3^2 - 2P\cdot p_3 \notag\\
  &= M^2 + m_3^2 - 2ME_3
\end{align}
がいえる.
ここで,
\begin{equation}
  E_3 = \sqrt{\vec{p}_3^2 + m_3^2} \ge m_3
\end{equation}
より,
\begin{equation}
  m_{12}^2 \le M^2 + m_3^2 - 2Mm_3 = \left(M - m_3\right)^2
\end{equation}
が分かる.
一方,
\begin{align}
  m_{12}^2 &= \left(p_1 + p_2\right)^2 \notag\\
  &= p_1^2 + p_2^2 + 2p_1\cdot p_2 \notag\\
  &= m_1^2 + m_2^2 + 2p_1\cdot p_2 
\end{align}
であり,
\begin{align}
  p_1 \cdot p_2 &= E_1E_2 - \vec{p}_1 \cdot \vec{p}_2 \notag\\
  &\ge E_1E_2 - \left| p_1 \right| \left| p_2 \right| \notag\\
  &= E_1E_2 - \sqrt{E_1^2 - m_1^2} \sqrt{E_2^2 - m_2^2} \notag\\
  &\ge E_1E_2 - E_1E_2 + m_1m_2 \notag\\
  &= m_1m_2
\end{align}
が成り立つので,
\begin{equation}
  m_{12}^2 \ge m_1^2 + m_2^2 + 2m_1m_2 = \left(m_1 + m_2\right)^2
\end{equation}
がいえる.
以上より,
\begin{equation}
  \left(m_1 + m_2\right)^2\le m_{12}^2 \le \left(M - m_3\right)^2 
\end{equation}
が成り立つ.
同様にして,
\begin{equation}
  \left(m_2 + m_3\right)^2\le m_{23}^2 \le \left(M - m_1\right)^2 
\end{equation}
もいえる.

\subsection*{2.3}
\begin{equation}
  E_2^* = \frac{m_{12}^2 - m_1^2 + m_2^2}{2m_{12}},\qquad E_3^* = \frac{M^2 - m_{12}^2 + m_3^2}{2m_{12}}
\end{equation}
を用いると,
\begin{align}
  m_1^2 &= -2m_{12}E_2^* + m_{12}^2 + m_2^2 \\
  M^2 &= 2m_{12} E_3^* + m_{12}^2 - m_3^2
\end{align}
が分かるので,前問より
\begin{align}
  (m_2+m_3)^2 \le m_{23}^2 &\le M^2 - 2Mm_1 + m_1^2 \notag\\
  &= 2m_{12}(E_3^*-E_2^*+m_{12}) + m_2^2 - m_3^2 \notag\\
  &\qquad - 2\sqrt{(-2m_{12}E_2^* + m_{12}^2 + m_2^2)(2m_{12} E_3^* + m_{12}^2 - m_3^2)}
\end{align}
と書ける.

\section*{3}
\subsection*{3.1}
運動量保存則より,$\pi^0$と$n$の運動量はそれぞれ$\vec{p}, -\vec{p}$と書ける.
これより,それぞれのエネルギーは
\begin{align}
  E_{\pi^0}^2 &= \left|\vec{p}\right|^2 + m_{\pi^0}^2 \\
  E_{n}^2 &= \left|\vec{p}\right|^2 + m_{n}^2 
\end{align}
となる.
辺々引いて,エネルギー保存則
\begin{equation}
  E_{\pi^0} + E_{n} = m_{\pi^-} + m_{p}
\end{equation}
を用いると,
\begin{equation}
  E_{\pi^0} - E_{n} = \frac{m_{\pi^0}^2-m_{n}^2}{m_{\pi^-} + m_{p}}
\end{equation}
がいえる.したがって,これとエネルギー保存則から
\begin{equation}
  E_{\pi^0} = \frac{1}{2} \left(m_{\pi^-} + m_{p} + \frac{m_{\pi^0}^2-m_{n}^2}{m_{\pi^-} + m_{p}}\right)
\end{equation}
がいえる.
よって,
\begin{equation}
  E_{\pi^0} = \frac{m_{\pi^0}}{\sqrt{1-\beta^2}}
\end{equation}
と書けることを用いると,
\begin{equation}
  \beta = \sqrt{1- \left(\frac{2m_{\pi^0}(m_{\pi^-}+m_p)}{(m_{\pi^0}+m_p)^2+(m_{\pi^0}^2-m_n^2)}\right)^2} \approx 0.194
\end{equation}
と求まる.

\subsection*{3.2}
$\pi^0$が崩壊してできる2つの光子のエネルギーと運動量をそれぞれ$E_i, \vec{p}_i\,(i=1,2)$とおくと,光子の質量がゼロであることとエネルギー・運動量保存則から
\begin{align}
  E_{\pi^0} &= E_1 + E_2 = \left|\vec{p}_1\right| + \left|\vec{p}_2\right|\\
  \vec{p} &= \vec{p}_1 + \vec{p}_2
\end{align}
が成り立つ.
ここで,$\vec{p}$と$\vec{p}_1$のなす角を$\theta$として
\begin{equation}
  \vec{p}\cdot \vec{p}_1 =   \left|\vec{p}\right|\left|\vec{p}_1\right|\cos{\theta}
\end{equation}
が成り立つとする.
このとき,
\begin{align}
  m_{\pi^0}^2 &= E_{\pi^0}^2 - \vec{p}^2 \notag\\
  &= 2\left(\left|\vec{p}_1\right|\left|\vec{p}_2\right| - \vec{p}_1\cdot \vec{p}_2\right) \notag\\
  &= 2\left(\left|\vec{p}_1\right|\left(E_{\pi^0} - \left|\vec{p}_1\right|\right) - \vec{p}_1\cdot \left(\vec{p}-\vec{p}_1\right)\right) \notag\\
  &= 2\left|\vec{p}_1\right| \left(E_{\pi^0} - \left|\vec{p}\right|\cos{\theta}\right) \notag\\
  &= 2E_1 \left(E_{\pi^0} - \left|\vec{p}\right|\cos{\theta}\right)
\end{align}
より,
\begin{equation}
  E_1 = \frac{m_{\pi^0}^2}{2(E_{\pi^0}-\left|\vec{p}\right|\cos{\theta})}
\end{equation}
と求まる.
したがって,
\begin{equation}
  \frac{m_{\pi^0}^2}{2(E_{\pi^0} + \left|\vec{p}\right|)}\le E_1 \le \frac{m_{\pi^0}^2}{2(E_{\pi^0} - \left|\vec{p}\right|)}
\end{equation}
がいえる.
これを$\beta$を用いて書き直すと,
\begin{equation}
  \frac{m_{\pi^0}\sqrt{1-\beta^2}}{2(1+\beta)}\le E_1 \le \frac{m_{\pi^0}\sqrt{1-\beta^2}}{2(1-\beta)}
\end{equation}
となるので,値を代入すると,
\begin{equation}
  \SI{55.4}{\MeV} \le E_1 \le \SI{82.2}{\MeV}
\end{equation}
となる.