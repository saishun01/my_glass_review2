\documentclass[a4paper,11pt]{jsarticle}


% 数式
\usepackage{amsmath,amsfonts}
\usepackage{bm}
\usepackage{physics}
% 画像
\usepackage[dvipdfmx]{graphicx}
% ローマ数字
\usepackage{otf}
% 単位
\usepackage{siunitx}
% 表
\usepackage{multirow}
% 化学反応
\usepackage[version=4]{mhchem}


\begin{document}

\title{機械学習概論 第2回レポート}
\author{05-211525 齋藤駿一}
\date{\today}
\maketitle

\section{前処理}
まず,訓練データ(dry\_bean\_train\_data.csv)およびテストデータ(dry\_bean\_test\_data.csv)の要素を次のように前処理した.

\begin{enumerate}
  \item 17個の特徴量の数値をそれぞれ,平均0,分散1となるように変換した.
  \item 7つの種類を0から6までの整数値に変換した.
\end{enumerate}

\section{サポートベクトルマシンによる学習}

まず,訓練データを用いてサポートベクトルマシン(SVM)の学習を行い,テストデータの特徴量からその種類を推測させた.
その際,SVMのカーネルは線形カーネルとした.
また,コストパラメータ$C$とカーネル幅$\gamma$は$10^{a}\,(a=-3,-2,-1,0,1,2)$の中から選び,訓練データを用いて10-foldの交差検証を行い,最適と判断された量を採用した.

その結果,交差検証での正答率は92.4\%,テストデータでの正答率は93.1\%となった.

\section{多層ニューラルネットによる学習}

次に,SVMのかわりに多層ニューラルネットを用いて学習を行った.
前処理として,訓練データとテストデータにおいて,7つの種類をone-hot codingに直した.
ここでは,以下ようにニューラルネットを構築した.
まず,入力ユニット16個(特徴量の数と同じ)を30個の中間ユニットに全結合させ,その活性化関数をtahnとした.
次に,その中間ユニットを7個(種類の数と同じ)の出力ユニットに全結合させ,その活性化関数をsoftmaxとした.
また,損失関数は交差エントロピーとし,最適化アルゴリズムはSGDとした.
そして,これをバッチサイズ10で300エポックの間学習させた.

その結果,最終的な正答率は,訓練データで94.4\%,テストデータで93.6\%となった.

その後,バッチを正規化したり,層の数を増やしたり,Early stopping(訓練データを9:1に分割して前者を学習に用い,後者を過学習する前に学習を止めるために用いる)をしたりすることで,精度の向上を試みた.
しかし,訓練データを減らしたことによる精度の低下のため,上述の精度を超えられなかった.

また,ガウス過程を用いて学習させることも試したが,こちらは計算時間が非常に長くなってしまい,断念した.

\end{document}